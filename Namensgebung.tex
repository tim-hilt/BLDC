\section{Namensgebung und Abgrenzung zum bürstenbehafteten Gleichstrommotor}

Die englische Bezeichnung \glqq{}BLDC\grqq{} (Brushless-Direct-Current-Motor) bedeutet im Deutschen \glqq{}bürstenloser Gleichstrommotor\grqq{}. Um zu verstehen, weshalb der von uns beschriebene Motortyp diesen Namen trägt wollen wir zunächst auf seinen Vorläufer --- den bürstenbehafteten Gleichstrommotor eingehen und die beiden Motorentypen im Verlauf der Arbeit weiter voneinander abgrenzen.

Bürstenbehaftete Gleichstrommotoren werden auch als \emph{mechanisch kommutierte Gleichstrommotoren} bezeichnet. Diese \glqq{}werden als permanenterregte Nebenschlussmotoren ausgelegt. Sie zeichnen sich durch einen linearen Strom–Drehmoment Verlauf aus, der von der Winkellage des Rotors nahezu unabhängig ist\grqq{} \parencite[S.51]{Probst2011}. Hier werden mehrere Eigenschaften dieses Motortyps deutlich. Der bürstenbehaftete Gleichstrommotor ist mechanisch kommutiert. Das bedeutet, dass durch Kontaktierung zweier leitender Materialien Strom durch die Motorwicklungen fließen und der Motor sich so drehen kann.

\begin{figure}[H]
  \centering
  \includegraphics[width=8cm]{./Grafiken/AufbauBürstenbehafteterGleichstrommotor}
  \caption[Aufbau bürstenbehafteter Gleichstrommotor]{Prinzipaufbau des bürstenbehafteten Gleichstrommotors (Quelle: \parencite[S.51]{Probst2011})}
  \label{fig:AufbauBürstenbehaftet}
\end{figure}

Dieser Aufbau hat einige Vorteile. So ist zum Beispiel die Steuerung der Drehzahl und -richtung des Motors durch Variation der Ankerspannung sehr einfach zu realisieren. Auch regelungstechnisch bietet dieser Motortyp aufgrund des linearen Strom-Drehmoment-Verhaltens große Vorteile. Die Herstellung des Motors ist aufgrund insbesondere dieser beiden Eigenschaften sehr kostengünstig umzusetzen. \glqq{}Große Drehzahlstellbereiche bis zu 1:10000 und eine hohe Gleichlaufgüte\grqq{} \parencite[S.51]{Probst2011} sind weitere positive Eigenschaften, die dem bürstenbehafteten Gleichstrommotor eine klare Daseinsberechtigung bescheinigen.

Der in Abbildung \ref{fig:AufbauBürstenbehaftet} dargestellte Aufbau des bürstenbehafteten Gleichstrommotors zeigt im unteren und oberen Bereich jeweils eine Kohlebürste. Diese aus Graphit bestehenden Bauteile werden mithilfe einer Feder durchgehend auf den Kommutator gedrückt. Beginnt Strom zu fließen, so beginnt sich der Motor zu drehen. Da die Bürsten nach wie vor auf den Kommutator drücken entsteht Reibung und infolgedessen eine mechanische Abnutzung.

\glqq{}Im  Vergleich  zu  anderen  Motorkonzepten  ergeben  sich  die  wesentlichen negativen Eigenschaften des bürstenbehafteten Motors aus seinem prinzipiellen Aufbau\grqq{} \parencite[S.52]{Probst2011}. Diese negativen Eigenschaften beinhalten zum einen die mechanische Abnutzung und die daraus resultierende Notwendigkeit verschlissene Bürsten zu ersetzen, als auch Reibungsverluste und Probleme mit der Abführung von Abwärme, die durch die mechanische Kommutierung entstehen. Hinzu kommen Nachteile durch unvermeidbare Übergangswiderstände von den Kohlebürsten zum Kommutator. Dadurch wird der maximal mögliche Ankerstrom im Stillstand (Durchbrennen), als auch bei hohen Drehzahlen (Bürstenfeuer) begrenzt \parencite[vgl.][S.52]{Probst2011}. Diese Effekte können insbesondere im explosionsgefährdeten Arbeitsumfeld lebensbedrohlich werden.

Trotz dieser Nachteile wurde der bürstenbehaftete Gleichstrommotor nicht vollständig durch andere Motorenkonzepte verdrängt, denn insbesondere bei Applikationen mit kostenkritischen Anforderungen und wenig mechanischer Belastung können seine Vorteile hervorragend genutzt werden.

Im Folgenden wird nun die Funktionsweise unterschiedlicher \emph{bürstenloser} Gleichstrommotorentypen erklärt

%%% Local Variables:
%%% mode: latex
%%% TeX-master: "BLDC"
%%% End:
